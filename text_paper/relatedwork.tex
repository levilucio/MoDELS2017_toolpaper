\section{Related Work}
\vspace{-.4cm}Given the recent fast-paced development of Artificial Intelligence
relying on increasingly powerful hardware, a number of projects have devoted
effort to the generation of controllers from requirements. The ARSENAL
project~\cite{ghosh2016arsenal} has as starting point specifications written in
arbitrary natural language and uses the GR-1~\cite{piterman2006synthesis}
synthesizer for automatically building controllers. In~\cite{YanCC15} the
authors also use the GR-1 synthesizer to automatically build robot
controllers. The work of Yan et. al.~\cite{YanCC15} takes as inputs full LTL
specifications and includes features such as the use of dictionaries for
automatically derive relations between terms, or guessing the I/O partitioning
that allow detecting inconsistencies in the specifications. The commercial
argosym STIMULUS tool~\cite{jeannet16}, while not based on AI algorithms from controller
synthesis, is a commercial platform that allows specifying requirements in a
formal language using a close-to natural language syntax. Requirements expressed
in STIMULUS can be simulated and test cases can also be directly generated from
the requirements.

Our approach differs from the GR-1-based projects mentioned above in the sense
that we do not aim at applying pure natural language parsing to arbitrary
requirements. Using EARS allows us to provide the readability of the English
language while gently contraining it to fit the domain of requirements
gathering. Also, rather than using the full expressiveness of LTL, we have
restricted our approach to the \textsf{GXW} subset of LTL which is handled by
the \textsf{autoCode4} tool. Using this subset it is possible to directly
generate controllers as SDF diagrams, which are easy to inspect and to simulate.
Tools that are based on GR-1 or bounded synthesis~\cite{schewe2007bounded}
typically produce controllers as BDD or explicit state machine structures that
can be very large and difficult to inspect or simulate.

Regarding the STIMULUS tool, our approach was conceptually though of starting
from an opposite direction -- while STIMULUS essencially uses as central
formalism state machines wrapped by a syntactic-sugar English-like specification
language, \textsf{EARS-CTRL} uses a constrained version of the English language.
We have purposefully placed EARS at the center on our tool -- the goal has been
to adapt the subset of LTL used by \textsf{autoCode4} to EARS and to remain
unbiased towards the formalisms ``under the hood''. Unlike in our work, STIMULUS
relies on the state machines underlying the requirements to allow simulation as
in fact the approach's goal is to verify requirements and not to synthesize
usable controllers.\vspace{-.5cm}

