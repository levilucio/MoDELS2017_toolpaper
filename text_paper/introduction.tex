\section{Introduction}
\vspace{-.2cm}The ultimate goal in human-computer interaction is that humans can
``explain'' to computers their needs, using human-centered languages. Computers
would then automatically perform the actions that satisfy those needs. This
trend is on the rise with automated call-centers or personal
assistants that, through voice commands, can search for itineraries,
restaurants, hotels and even perform online bookings.

In this paper we describe the \textsf{EARS-CTRL} tool for building and verifying
software controllers. \textsf{EARS-CTRL} has as starting point the EARS (Easy
Approach to Requirements Syntax) language. EARS was created at Rolls-Royce to
improve the expression of natural language requirements~\cite{EARS09} and can be
seen as a way to ``gently'' constrain English. The application of EARS produces
requirements in a small number of patterns. EARS copes well with large
specifications of requirements for several domains~\cite{EARS10,EARS16}. EARS is
also an effective way of reducing many of the problems that plague requirements
documents written using unconstrained natural language~\cite{EARS09}.

With the \textsf{EARS-CTRL} tool we make a step in the direction of controller
construction using natural language as a central specification artifact.
After specifying the vocabulary to be used in the specification, a requirements
engineer writes the specification using EARS templates. Then, at the press of a
button, the controller is synthesized. Simulation and test case
generation panels allow the requirements engineer to immediately
experiment with and validate the controller.

This paper builds on a previous article~\cite{LucioRCM17}.
Our new contributions are a revision of the requirements language of
\textsf{EARS-CTRL}, which is now fully aligned with the original EARS. We do so
by improving the coverage of the semantic gap between EARS and the underlying
logical formalism used by the controller synthesizer. We now also offer the
possibilities of simulating requirements specifications as well as of generating test cases.
The \textsf{EARS-CTRL} tool is freely available as a GitHub project
at~\cite{EARSProject}. \vspace{-.4cm}

% \levi{Talk about the rise of AI in program synthesis and formal methods and how
% we enable requirements engineers to get closer to get closer to a code-free
% program synthesis by providing an appropriate IDE} 
% 
% Note that the front-end of
% the tool is based on the MPS (Meta-Programming System) meta-editor.

% The contributions of this paper are:
% \begin{itemize}
%   \item EARS syntax:
% \begin{itemize}
%   \item components are now devices that can have sensors and actuators and are
%   described in the glossary. This makes it for a more fluid English description. 
%   \item until clauses have been removed from templates. 
% \end{itemize}
% \item{Synthesis}
% \begin{itemize}
%   \item lifting of error codes from the synthesizer. 
% \end{itemize}
% \item Simulation and test case generation.
% \begin{itemize}
%   \item interaction with simulink for simulation.
%   \item interaction with simulink for text case generation
% \end{itemize}

%\end{itemize}