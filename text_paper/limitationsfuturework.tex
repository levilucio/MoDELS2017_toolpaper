\section{Limitations and Future Work}
\label{sec:limfuturework}
\vspace{-.4cm}Due to the boolean representation of sensors and actuators in
\textsf{autoConf4} it is currently not possible for \textsf{EARS-CTRL} to express or analyze
states of the system that involve numerical data. This naturally limits our
approach to controllers for systems where sensors or actuators can be
represented using boolean types.
For instance, expressing a state such as ``the throttle is pushed to 1/4 of its
capacity'' using \textsf{EARS-CTRL} would at best involve modelling four
different sensors in order to partition the input space of a single sensor. Such
an approach may prove to be infeasible and/or impractical and other code
synthesizers for \textsf{EARS-CTRL} might be considered in the future.\levi{which ones?}

Future work will concentrate on exploring the usability of
\textsf{EARS-CTRL} for larger case studies. In particular we expect to continue
the collaboration with Rolls-Royce in order obtain real requirements such that
the synthesis and verification mechanisms explained in this paper can be put to
the test in the field. It is of particular interest to understand not only how
\textsf{EARS-CTRL} \textsf{autoConf4}-based synthesis will scale, but also how
the verification and debug mechanisms we propose are helpful in practice.\levi{is this too self-deprecating?}
