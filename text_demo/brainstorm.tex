More points:
4. Code generation and synthesizer for EARS-based requirements
5. Test case generation for conformance checking of the generated controller
6. Interfacing with Matlab Simulink
5. Viewing the Results
7. Lessons learned
8. Start discussing the steps as flow model and follow exactly the same
steps!!!! 
8.1 Expression of Controller Requirements as EARS and related Models (e.g.,
Glossary) 
8.2 MPS constraints to ensure completeness of the EARS-based requirements
(hint: discuss some examples if something breaks) 
8.3 Test case generation process (step-by-step) that includes interfacing with
simulink, test case generation sequences, showing the results of the results as
inputs and output as the SimulinkResult model
\emph{Notes:}
Point can be arised that if you are generating code from the model why do we need tests? we can sell our
idea by stating that we don't necessarly want to generate the code but write the
controller requirements in EARS and somebody else can write the code. Some
engineers don't rely on automatic code generation and want to develop/build
controllers explicitly. In such a situation, test case generation would help to
perform conformance between the specified controller and its respective
implementation. 
dots
\emph{Notes:} 
Point can be arised that if you are generating code from the model why do we need tests? we can sell our
idea by stating that we don't necessarly want to generate the code but write the
controller requirements as EARS and somebody else can write the code. Some
engineers don't rely on automatic code generation and want to develop/build
controllers explicitly. In such a situation, test case generation would help to
perform conformance between the specified controller and its respective implementation.