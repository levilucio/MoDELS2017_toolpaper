\vspace{-.5cm}
\section{Tool Demonstration}
\label{sec:demo}
\vspace{-.3cm}
In this Section, we discuss the main steps of process that needs to be followed
for synthezing and performing the conformance\levi{``conformance'' has strict
meanings in different contexts. I would say ``check if the controller behaves
as expected''} analysis.
\vspace{-.3cm}
\subsection{Glossary building and terms definition}
\vspace{-.2cm}
The first step towards writing the controller specifications as a natural
language in \textsf{EARS-CTRL} is to define the glossary terms. 
The user is provided with an editor
(figure~\ref{fig:glossary_def}) to define glossary terms for the following,
1) components that interface with the controller, 2) sensors and actuators those components make available to the controller and 3) rules between
signals definitions as aliases and invariants.
\vspace{-.2cm}
\begin{figure*}[!h]
\centering
\includegraphics[width=1\textwidth]{./images/QC_Glossary_Def.png}
\caption{Step-by-step glossary building for a Quiz Controller: (\emph{A})
Components definition, (\emph{B}) Sensors definition, (\emph{C}) Actuators
definition and (\emph{D}) Completed QC Glossary}
\label{fig:glossary_def}
\end{figure*}
\vspace{-.2cm}
\subsection{Building \textsf{EARS-CTRL} requirements for the Quiz Controller}
\vspace{-.2cm}
The user is provided with an editor built in MPS to specify the
controller behavior. The user can input the requirements simply by filling in the instance of the EARS template with
placeholders in the editor. Figure~\ref{fig:EARS_req} provides step-by-step
guidance to write the EARS requirements. In the projectional editor the user is
provided with the option (i.e., using CTRL+Space keys on keyboard) to intantiate
the EARS-based template with placeholders (\emph{Step A}). After obtaining an
instance of the EARS template, the user starts filling in the placeholders with the information
(i.e., glossary definitions) (\emph{Step B}). \emph{Step C} of the
figure~\ref{fig:EARS_req} depicts a complete Quiz Controller specification.
\begin{figure*}[!h]
\centering
\includegraphics[width=1.2\textwidth]{./images/Req_Spec_Steps.png}
\caption{Step-by-step \textsf{EARS-CTRL} for controller requirements: (\emph{A})
Empty instance of EARS template with placeholders, (\emph{B}) Filling instance of an EARS
template with information and (\emph{C}) Completed EARS Specification }
\label{fig:EARS_req}
\end{figure*}
\vspace{-.2cm}
\subsection{Synthesizing \textsf{EARS-CTRL} requirements}
\vspace{-.2cm}
The user of the tool can attempt to synthesize the controller
once the \levi{requirements for} controller is completely specified. In order to
synthesize the controller, the user can apply the \emph{Transform} intention by using the
\emph{Alt+Enter} keys on the root of the specification (as shown in \emph{part
A} of figure~\ref{fig:Spec_transform}). The generated outputs after applying the
intention comprised of, 1) \emph{Controller Holder}: Data Flow
Diagram with blocks \levi{and wires} (\emph{part B} of
figure~\ref{fig:Spec_transform}) and 2) \emph{Gate Descriptors}:
Pseudo code representing the behavior of the blocks (\emph{part C} of
figure~\ref{fig:Spec_transform}).
%\vspace{-.7cm}
\begin{figure*}[!h]
\centering
\includegraphics[width=1\textwidth]{./images/Transform.png}
\caption{Controller generation steps: (\emph{A}) Applying Intention (Alt+Enter),
(\emph{B}) Data Flow Diagram of the controller (an excerpt) and (\emph{C})
Pseudo code representing the behavior of the blocks (an excerpt)}
\label{fig:Spec_transform}
\end{figure*}
\vspace{-.5cm}
\subsection{Simulation and Test Cases for Conformance Analysis}
\vspace{-.2cm}
Conformance analysis for verification of the generated controller with respect
to its specification \levi{not to its specification, but to the expected
behavior} can be performed by the following techniques, 1) simulating the
behavior with Simulink engine \cite{MatlabSimulink} as a back-end and 2) automatically generating the test cases.
The resulted traces are analyzed in order to check if the controller behaves as
expected. \levi{you can remove this following sentence}The analysis can be
performed as follows,
%\vspace{-.5cm}
\subsubsection{Simulation}
\vspace{-.2cm}
In order to perform simulation of the generated controller, the user needs to
provide the path in the \emph{Controller Holder} model to the folder containing
the Simulink blocks (i.e., .m files)\levi{you can be more generic here, just say
that we have implemented Simulink blocks to allow for simulation, or just remove
the sentence altogether}.
More information can be found at the Github project page \cite{EARSProject}. The
user than applies the \emph{GenerateSimulinkModel} intention on the root of the
\emph{Controller Holder} (i.e., Step A in figure~\ref{fig:SimulationSteps}).
After getting the Simulink model generated, the user can apply the intention
\emph{AddOutputChecker} (i.e., Steps B and C in figure
~\ref{fig:SimulationSteps}) to start simulating the controller behavior by
providing manual input. For simulation purpose, the user gets the
\textsf{EARS-CTRL} panel \levi{why do we need to launch the intention AND
create the panel?} that allows the user to simulate the controller by providing
a sequence of inputs manually.
Outputs are incrementally added to the panel as new inputs are provided by the
requirements engineer (i.e., Step D in figure~\ref{fig:SimulationSteps}).
A \textsf{“Reset Results”} button enables the user to reset the controller to
its initial state.
\begin{figure*}[!h]
\centering
\includegraphics[width=1\textwidth]{./images/Simulation_Steps.png}
\caption{Simulation steps for verification: (\emph{A}) Applying
GenerateSimulinkModel Intention, (\emph{B}) Applying intention AddOutputChecker,
(\emph{C}) Generated Empty \textsf{EARS-CTRL} panel and (\emph{D}) Manually
simulating the Controller}
\label{fig:SimulationSteps}
\end{figure*}
\vspace{-.3cm}
\subsubsection{Automatic Test cases Generation} 
\vspace{-.5cm}
The user can automatically generate test cases
(figure~\ref{fig:TestCaseGeneration}) by setting in the testing parameters.
For the test case generation, the user inputs the
parameters in the \textsf{EARS-CTRL} panel.
The parameters required for test case generation are as follows, 1) \textsf{Test
Sequence Length(integer value)}, 2) \textsf{Allow parallel inputs(true/false)} and 3) \textsf{Allow repeated inputs (true/false)} .
