\vspace{-.2cm}
\section{Tool Demonstration}
\label{sec:demo}
\subsection{Building a Glossary}
\vspace{-.1cm}
The first step towards writing the controller specifications in natural
language using \textsf{EARS-CTRL} is to define a glossary of terms. 
As is depicted in figure~\ref{fig:glossary_def}, the user is provided with an
editor to define glossary terms for: components that interface
with the controller; sensors and actuators those components make
available to the controller; invariant relations that should hold
between the sensor and actuator signals; and for ease of writing,
aliases to formulas involving sensors or actuators. During the demonstration we
will start from a glossary that is mostly built and will add the missing
components.
 
\vspace{-.2cm}
\begin{figure*}[!h]
\centering
\includegraphics[width=1\textwidth]{./images/QC_Glossary_Def.png}
\caption{Step-by-step glossary building for a Quiz Controller: \emph{(A)}
components definition, \emph{(B)} sensors definition, \emph{(C)} actuators
definition and \emph{(D)} completed QC glossary}
\vspace{-.8cm}
\label{fig:glossary_def}
\end{figure*}
\subsection{Building \textsf{EARS-CTRL} requirements for the Quiz Controller}
\vspace{-.2cm}
Figure~\ref{fig:EARS_req} provides the set of steps required to write a set of
EARS requirements. In the projectional editor the user presses the CTRL+Space
key combination to instantiate an EARS-based template that presents a number of
placeholders (A). After obtaining an instance of the EARS template, the user
fills in the placeholders with information coming from the glossary definitions
(B). (C) depicts the complete Quiz Controller specification. As for the
glossary, during the demo we will use a mostly complete set of requirements and
will add the missing ones.
\begin{figure*}[!h]
\centering
\includegraphics[width=1.2\textwidth]{./images/Req_Spec_Steps.png}
\caption{Step-by-step guidance for building controller requirements in
\textsf{EARS-CTRL}, (\emph{A}) empty instance of EARS template with placeholders, (\emph{B}) filling instance
of an EARS template with information and (\emph{C}) completed EARS Specification}
\label{fig:EARS_req}
\vspace{-.4cm}
\end{figure*}
\vspace{-.4cm}
\subsection{Synthesizing \textsf{EARS-CTRL} requirements}
\label{SynthReq}
\vspace{-.1cm}
Once the requirements for controller are completely specified, we will
synthesize the controller. For that the \emph{Transform} intention is used
by using the \emph{Alt+Enter} keys on the root of the specification (as shown in
(A) of figure~\ref{fig:Spec_transform}). The generated output after applying the
intention is comprised of: (B) the \emph{Synchronized Data Flow} (SDF) diagram
containing blocks connected by wires; (C) pseudo code representing the behavior
of each of the blocks used by the synthesized controller; (D) a Simulink block
diagram; and (E) empty panel for simulation and test case generation.
%\vspace{-.7cm}
\begin{figure*}[!h]
\centering
\includegraphics[width=1\textwidth]{./images/Transform.png}
\caption{Controller generation steps: (\emph{A}) applying intention \emph{Alt+Enter},
(\emph{B}) synchronized DFD of the controller (an excerpt), (\emph{C}) pseudo
code representing the behavior of the blocks (an excerpt), (\emph{D})
generated Simulink model (an excerpt) and (\emph{E})
generated empty panel for simulation}
\label{fig:Spec_transform}
\end{figure*}
\vspace{-.2cm}
\subsection{Simulation and Test Cases for Validation of Controller
Behavior}
\vspace{-.1cm}
Validation of the generated controller can be done using simulation and/or
test case generation.
\vspace{-.3cm}
\subsubsection{Simulation}
\vspace{-.2cm}
Part (A) of Figure~\ref{fig:PanelView} depicts the
step-by-step view of the simulation and generation panel. In order to do
step-by-step simulation we will perform the following steps: 1) select the
\textsf{\emph{stepView}} from the drop-down menu on the panel; 2) select or
unselect the radio buttons to set the sensors to \textsf{ON} or \textsf{OFF}
respectively; and 3) press the execute button.
The controller's output is added to the lower part of the panel.
\begin{figure*}[!h]
\centering
\includegraphics[width=.9\textwidth]{./images/Two_Views_Panel.png}
\caption{Simulation and test case generation panels, A) step-by-step simulation
view and B) test case generation view}
\label{fig:PanelView}
\vspace{-.2cm}
\end{figure*}
\vspace{-.2cm}
\subsubsection{Debugging Using Step-by-Step Simulation}
In order to debug quiz controller we previously generated we will perform a few
simulation steps: we first select the following inputs on the panel:
\emph{pupil Button 0} is set to \textsf{ON} and the remaining signals are set to
\textsf{OFF}.
The generated outputs reveal that the pupil indicator is in the \textsf{ON}
state which is the intended behavior. We then set the \emph{reset button
is pressed} to \textsf{ON} while the rest of the input signals are
set to \textsf{OFF}.
We expect that the controller's outputs are all set to the \textsf{OFF} state
in response to the reset, however the controller sets the indicator for the
pupil to the \textsf{ON} state as shown in Figure~\ref{fig:simError}. This investigation leads us to identify
an error in ``Req6'' of Figure~\ref{fig:QC_reqs}, where the pupil indicator is mistakenly turned on.
 \begin{figure*}[!h]
\centering
\includegraphics[width=1\textwidth]{./images/Simulation_Error.png}
\caption{Error identified during step-by-step simulation}
\label{fig:simError}
\vspace{-.6cm}
\end{figure*}
%\vspace{-.3cm}
\subsubsection{Automatic Test Cases Generation} 
\vspace{-.5cm}
Due to lack of space in this document, we are not able to present in a
screenshot a set results of test case generation for presentation at the
demonstration. However, during the demo we will exemplify our tool's test
generation capabilities by creating test set based on the available test generation parameters: test case length, repetition of
inputs in the test case allowed / not allowed and more than one sensor turned on
per input vector allowed / not allowed.
