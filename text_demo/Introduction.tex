\section{Introduction}
\label{sec:intro}
EARS (Easy Approach to Requirements Syntax) is an effective technique utilized
by many organizations (e.g., Rolce Royce and others) to build effective
requirements \cite{EARS}. In this demonstration part of the paper, we are going
to go through steps for building the EARs-based requirements and performing test case generation for the
sliding door example of a PLC based controller. Our tool is developed in
Jetbrains MPS, a projectional meta editor for DSL
development\cite{DBLP:conf/pppj/PechSV13}. The work presented here is an
extension to our previous work on building complex controller requirements and
automatic generation of the specified requirements
\cite{LucioRCA16}.

A brief installing information and links to the main github projects, URL links
of the projects.

Discuss the main purposes of our work,that are,
\begin{itemize}
  \item Progressively build a set of requirements for the controller
  \item Perform analysis to ensure correct building of the example case
  (whether the controller is synthesizable or not!!!) and correctness by
  construction\ldots
  \item Automatic realisation of the synthesized controller (one click approach)
  \item Automatic test case generation for performing conformance analysis
\end{itemize}
\emph{Notes:}
Point can be arised that if you are generating code from the model why do we need tests? we can sell our
idea by stating that we don't necessarly want to generate the code but write the
controller requirements in EARS and somebody else can write the code. Some
engineers don't rely on automatic code generation and want to develop/build
controllers explicitly. In such a situation, test case generation would help to
perform conformance between the specified controller and its respective
implementation. 
