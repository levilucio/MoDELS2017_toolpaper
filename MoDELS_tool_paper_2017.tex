% This is LLNCS.DEM the demonstration file of
% the LaTeX macro package from Springer-Verlag
% for Lecture Notes in Computer Science,
% version 2.4 for LaTeX2e as of 16. April 2010
%
\documentclass{llncs}
%

% Put edit comments in a really ugly standout display
\usepackage{xcolor} 
\usepackage{ifthen}
\usepackage{amssymb}
\usepackage{graphicx}
\usepackage{amsmath}
 
\newboolean{showcomments}
\setboolean{showcomments}{true} % toggle to show or hide comments
\ifthenelse{\boolean{showcomments}}
  {\newcommand{\nb}[2]{
    \fcolorbox{gray}{yellow}{\bfseries\sffamily\scriptsize#1}
    {$\blacktriangleright$#2$\blacktriangleleft$}
   }
   \newcommand{\version}{\emph{\scriptsize$-$working$-$}} 
  }
  {\newcommand{\nb}[2]{}
   \newcommand{\version}{} 
  }
 
\usepackage[linewidth=1pt]{mdframed}
\usepackage{lipsum} 

% for comments
\newcommand\levi[1]{\nb{Levi}{\textcolor{teal}{#1}}}
\newcommand\salman[1]{\nb{Salman}{\textcolor{blue}{#1}}}
\newcommand\ch[1]{\nb{Chih-Hong}{\textcolor{red}{#1}}}
\newcommand\mav[1]{\nb{Mav}{\textcolor{black}{#1}}}

\usepackage{makeidx}  % allows for indexgeneration
%
\begin{document}
%
\frontmatter          % for the preliminaries

\mainmatter              % start of the contributions
%
\title{EARS-CTRL: From Natural Language to Controller Code in 1
Click\levi{better title needed!}}
%
\titlerunning{}  % abbreviated title (for running head)
%                                     also used for the TOC unless
%                                     \toctitle is used
%
\author{Levi L\'ucio\inst{1} \and Salman Rahman\inst{1}
 \and Saad Bin Abid\inst{1} \and Alistair Mavin\inst{2}}
%
\authorrunning{} % abbreviated author list (for running head)
%
%%%% list of authors for the TOC (use if author list has to be modified)
\tocauthor{}
%
\institute{
fortiss GmbH\\
Guerickestra\ss e 25, 80805 M\"unchen\\
\email{\{lucio,abid\}@fortiss.org, salman.rahman@tum.de}\\
\and
Rolls-Royce, PO Box 31, Derby, UK\\
\email{alistair.mavin@rolls-royce.com}
}

\maketitle              % typeset the title of the contribution

\begin{abstract}
Abstract here\ldots
\end{abstract}

\section{Introduction}

The ultimate goal in human-computer interaction is that humans can ``explain''
to computers their needs, using human-centered languages (ideally natural
language). Computers will then perform the actions that will satisfy
those needs. This trend can be observed to be on the rise with automated
call-centers or personal assistants that, through voice commands, can
automatically search for itineraries, restaurants, hotels and even perform
online bookings.

In this paper we describe the \textsf{EARS-CTRL} tool for building and verifying
software controllers. \textsf{EARS-CTRL} has as starting point the EARS (Easy
Approach to Requirements Syntax) language which was created at
Rolls-Royce to improve gathering natural language requirements~\cite{EARS09}.
The language can be seen as ``gently'' constrained English. Through the use of a
small number of patterns, or formatted sentences, EARS copes well with large requirements
specifications for several domains~\cite{EARS10,EARS16}. It has additionally
been shown that using EARS is an effective way for reducing or, in some cases,
eliminating many of the problems that plague requirements documents written
using unconstrained natural language~\cite{EARS09}.

With the \textsf{EARS-CTRL} tool we attempt a step in the direction of
controller construction by using natural language as a central specification
artifact. After specifying the vocabulary to be used in the specification, a
requirements engineer writes the specification using EARS templates. Then, at
the press of a button the controller is synthesized. By using intuitive
simulation and test case generation panels the requirements engineer can
immediately experiment with and validate the controller by providing inputs and
observing the resulting outputs.

This paper is a follow-up of the article~\cite{LucioRCM17}, where we describe a
previous version of the \textsf{EARS-CTRL} tool.
Our new contributions are a revision of the requirements language of
\textsf{EARS-CTRL} which is fully aligned with the original EARS -- by better
covering the semantic gap between EARS and the underlying logical formalism for
controller synthesis. We now also offer the possibilities of simulating 
requirements specifications and of generating test cases.

The \textsf{EARS-CTRL} tool as well as a set of examples other than the ones we
present in this paper is freely available at a GitHub project at \ldots.

% \levi{Talk about the rise of AI in program synthesis and formal methods and how
% we enable requirements engineers to get closer to get closer to a code-free
% program synthesis by providing an appropriate IDE} 
% 
% Note that the front-end of
% the tool is based on the MPS (Meta-Programming System) meta-editor.

% The contributions of this paper are:
% \begin{itemize}
%   \item EARS syntax:
% \begin{itemize}
%   \item components are now devices that can have sensors and actuators and are
%   described in the glossary. This makes it for a more fluid English description. 
%   \item until clauses have been removed from templates. 
% \end{itemize}
% \item{Synthesis}
% \begin{itemize}
%   \item lifting of error codes from the synthesizer. 
% \end{itemize}
% \item Simulation and test case generation.
% \begin{itemize}
%   \item interaction with simulink for simulation.
%   \item interaction with simulink for text case generation
% \end{itemize}

%\end{itemize} 

\section{Highlights}

\subsection{``Real'' EARS}

A significant amount of effort was invested into making \textsf{EARS-CTRL} as close as
possible to the real EARS language used at Rolls-Royce. Given the fact that
EARS was not originally built to describe requirements at a level where they can
automatically be transposed into a real system, a large investment was made into
having a translation from EARS into the formalism used the controller
synthesizer we use in our work: the Linear Temporal Logic -- LTL -- formalism.
This translation was built to overcome as much as possible the barriers between the structured but non-formal nature of
EARS, and the strictly formal nature of LTL.

\begin{figure}[h!]
   \begin{center}
     \includegraphics[width=1\textwidth]{images/EARS-Reqs.png}
     \caption{\textsf{EARS-CTRL} Requirements for a sliding door controller}
     \label{fig:ears_reqs}
   \end{center}
 \end{figure}
 
Figure~\ref{fig:ears_reqs} ilustrates a set of \textsf{EARS-CTRL} requirements
for the software controller for a sliding door. By remaining as close as
possible to the original EARS syntax our editor allows building requirements as
correct English sentences that can easily read and understood by humans.

\begin{figure}[h!]
\begin{center}
\textbf{When} $\langle$trigger$\rangle$ \textbf{then the}
$\langle$system name$\rangle$ \textbf{shall}
$\langle$response$\rangle$ \textbf{until}
$\langle$trigger$\rangle$.
\caption{An non-standard state-driven EARS template including an \emph{until}
clause.}
\label{fig:ears_template_while}
\end{center}
\end{figure}

In~\cite{LucioRCM17} we have presented a previous version of \textsf{EARS-CTRL}
which included templates that, although not part of the original EARS, had been
introduced to simplify translation into LTL.  An example of one such templates
is presented in figure~\ref{fig:ears_template_while} -- the \textbf{while} segment of the
requirement is not standard EARS and it has been removed in the version of
\textsf{EARS-CTRL} we present in this paper. The work of briging the syntax of
\textsf{EARS-CTRL} closer to ``real'' EARS while preserving a semantically
meaniful translation into LTL was done with together with Alistair Mavin, the
author of this paper who first proposed EARS~\cite{EARS09}.
Our rationale is that, by remaining as faithful as possible to the original
EARS syntax, we: 1) benefit from all the advantages of using EARS already
investigated and described in the literature~\cite{EARS09,EARS16}; and 2)
provide to Rolls-Royce and potentially other companies a tool that can immediately be used by engineers trained in using EARS.

% , thus
% retaining its advantages of \ldots, as described in~\cite{}. In such as manner our tool benefits from all the research
% and advantages of EARS described in the literature~\cite{}.

% One of the main goals of our approach is to  In order to be able
% to ``write'' EARS-like requirements, a first analysis of the problem needs to be
% done in the shape of the creation of a of glossary for the problem at hand.
%
% 
% In order to reach a functioning controller we then allow interacting with the
% automatically generated synthesizer using Simulink. When necessary the engineer
% can edit the EARS requirements, where necessary, to be able to generate
% controllers exhibiting the expected behavior. This may come at the cost of
% sacrificing some of the readabilty of the original EARS
% requirements.\levi{blahblah}

\subsection{A Push-Button Approach}

Controllers can be fully automatically synthesized from EARS requirements into
controllers, at the push of a button. Note that in order to build requirements
model illustrated in figure~\ref{fig:ears_reqs} it is necessary to previously
build a glossary for the controller. Such a glossary identifies the components
of the system to be controlled. Each of those components contains sensors
and/or actuators the controller logic will use respectively as inputs from and
outputs to the real system. The vocabulary defined in the glossary is
appropriately proposed by the editor to fill in the placeholders when a new
requirement is built.

\begin{figure}[h!]
   \begin{center}
     \includegraphics[width=.5\textwidth]{images/glossary.png}
     \caption{\textsf{EARS-CTRL} Glossary for sliding door controller}
     \label{fig:ears_glossary}
   \end{center}
 \end{figure}
 
 We present in figure~\ref{fig:ears_glossary} the glossary for the automatic
 door controller specification. Note that the glossary allows defining
 invariants that will be taken into consideration during controller synthesis:
 for example in the last line of glossary in figure~\ref{fig:ears_glossary} we
 state that if the motor is \textsf{stopped}, then it cannot be
 \textsf{opening} or \textsf{closing}.

\subsection{Verification}

% Our \textsf{EARS-CTRL} tool provides a set of mechanisms for verification, in
% particular \emph{Well-Formedness by Construction} when the specification is
% being built and \emph{simulation} of the synthesized controllers by exercising the
% controller manually using \textsf{simulink} as a back-end. We can also generate
% test cases for controllers that can either be used to debug the controller by
% analysing traces of execution of the controller or to be used as test-cases for
% alternative implementations of the controllers which are not automatically
% synthesized.

\subsubsection{Well-Formedness by Construction}

Well-Formedness by construction is enforced at two levels in \textsf{EARS-CTRL}:
on the one hand only valid EARS requirement templates can be used. When the
requirements engineer selects an EARS template, the corresponding sentence is
shown to her as a structure with placeholders. These structures provide a
first level of well-formedness, as only correct sentences can be written. The
second level of well-formedness is guaranteed by the fact that only valid
sensors or actuators can be chosen in those sentences' placeholders. 

Well-formedness allows writing syntactically correct \textsf{EARS-CTRL}
specifications. The semantics of such specifications is then given by the
\textsf{autoConf4} synthesizer in the form of a synchronous dataflow diagram,
which \textsf{EARS-CTRL} can display graphically when synthesis is possible. In
the cases where synthesis is not possible the error code from the
\textsf{autoConf4} tool is lifted such that the requirements that prevent the controller from being
generated are pointed out. \levi{this is not done yet}

\subsubsection{Simulation}

\begin{figure}[h!]
   \begin{center}
     \includegraphics[width=.1\textwidth]{images/glossary.png}
     \caption{\textsf{EARS-CTRL} specification simulator}
     \label{fig:ears_glossary}
   \end{center}
 \end{figure}

Once a controller can be synthesized from a set of EARS requirements, it becomes
important to understand whether that controller behaves as expected. In order to
do so we have built a translator from synchronous dataflow diagrams into
\textsf{Simulink} block diagramns, which then allows us to exercise the behavior
of the controller. In figure~\ref{} we display the panel that allows
``playing'' the controller by providing a sequence of inputs manually. Outputs
are incrementally added to the panel as new inputs are provided by the
requirements engineer. Note that the controller keeps state, which means that
the order in which the commands are given matters. The ``Reset'' button in the
panel allows resetting the controller to its initial state.

\subsubsection{Generation of Test Cases}

\textsf{EARS-CTRL} allows generating test cases directly from the EARS
requirements. A test case consists of a sequence of $\langle
  input, output \rangle$ pairs, where each input is a vector of sensor states
  and each output a vector of actuator states. Note that individual sensors and
  actuators assume two states: ``0'': off; ``1'': on.

In the test generation panel the engineer can parameterize test generation by
providing the following values:
\begin{itemize}
  \item \emph{Test Sequence Length:} defines the maximum length of the $\langle
  input, output \rangle$ pair sequences to be generated.
  \item \emph{Allow parallel inputs:} enables or disables the possibility of
  having more than one sensor being active for inputs in the test case.
  \item \emph{Allow repeated inputs:} enables or disables having repeated inputs
  in the same test sequence. When enabled this parameter makes it such that
  an input vector cannot occur more than once in a test sequence (thus limiting
  the length of test sequences to the number of possible input vectors).
\end{itemize} 

\subsection{Code Generation}

Generation of C code can be achieved by running Simulink's code generator on the
Simulink model that is generated from the requirements EARS expressed in 
\textsf{EARS-CTRL}.


\section{Architecture}

\begin{figure}[t]
   \begin{center}
     \includegraphics[width=1\textwidth]{images/toolchain.png}
     \caption{The \textsf{EARS-CTRL} Tool Chain}
     \label{fig:ears_ctrl_toolchain}
   \end{center}
 \end{figure}
 
In figure~\ref{fig:ears_ctrl_toolchain} we depict the architecture of the
\textsf{EARS-CTRL} tool. In the following paragraphs we will provide to the
reader a brief description of the main components of the tool's architecture,
how those components have been implemented and which artifacts those
components interchange. 
 
\paragraph{1. EARS and Glossary Editor\\\\} 

As previously mentioned, an \textsf{EARS-CTRL} specification is composed of a
Glossary and of a set of requirements.

Additionally, in the glossary the requirements engineer can also express
relationships between signals such that necessary constrains can be defined, as show in figure~\ref{}.

Aliases for signals can also be defined, as can be seen in figure~\ref{}. These
aliases are helpful in making the specification readable, as can be seen in the
example in figure~\ref{}.

\paragraph{2. From EARS to Lineal Temporal Logic\\\\}

\paragraph{3. Synthesizing a Controller using \textsf{autoConf4}\\\\}

In order to achieve such as synthesis we
use the \textsf{autoConf4} tool, which takes as input as set of LTL and
propositional logic formulas. In figure~\ref{} we depict the translation of the
automatic door specification in figure~\ref{} into LTL. Note that the
translation is not directly one-to-one from EARS requirements into LTL formulas,
as some details are added by the translation algorithm in order to take into
consideration particularities of the \textsf{autoConf4} tool. This can be
observed \ldots.

\paragraph{4. Simulation and Test Case Generation Control Panel\\\\}

 In figure~\ref{} we display the panel that allows ``playing'' the
controller by providing a sequence of inputs manually. Outputs are incrementally
added to the panel as new inputs are provided by the requirements engineer.
Note that the controller keeps state, which means that the order in which the commands are given matters.
The ``Reset'' button in the panel allows resetting the controller to its initial
state.

Say here how the exploration mechanism works by simulating steps iteratively to
build the test cases according to the test set parameterization.

\paragraph{5. Simulation and Test Generation using Simulink\\\\}

Talk about how blocks from simulink are composed together to allow building the
expected behavior. Some blocks had to be purposefully build to allow simulation
to happran. This is done via a script that loads the model into simulink \ldots


\section{Related Work}
\vspace{-.4cm}Given the recent fast-paced development of Artificial Intelligence
relying on increasingly powerful hardware, a number of projects have devoted
effort to the generation of controllers from requirements. The ARSENAL
project~\cite{ghosh2016arsenal} has as starting point specifications written in
arbitrary natural language and uses the GR-1~\cite{piterman2006synthesis}
synthesizer for automatically building controllers. In~\cite{YanCC15} the
authors also use the GR-1 synthesizer to automatically build robot
controllers. The work of Yan et. al.~\cite{YanCC15} takes as inputs full LTL
specifications and includes features such as the use of dictionaries for
automatically derive relations between terms, or guessing the I/O partitioning
that allow detecting inconsistencies in the specifications. The commercial
argosym STIMULUS tool~\cite{jeannet16}, while not based on AI algorithms from controller
synthesis, is a commercial platform that allows specifying requirements in a
formal language using a close-to natural language syntax. Requirements expressed
in STIMULUS can be simulated and test cases can also be directly generated from
the requirements.

Our approach differs from the GR-1-based projects mentioned above in the sense
that we do not aim at applying pure natural language parsing to arbitrary
requirements. Using EARS allows us to provide the readability of the English
language while gently contraining it to fit the domain of requirements
gathering. Also, rather than using the full expressiveness of LTL, we have
restricted our approach to the \textsf{GXW} subset of LTL which is handled by
the \textsf{autoCode4} tool. Using this subset it is possible to directly
generate controllers as SDF diagrams, which are easy to inspect and to simulate.
Tools that are based on GR-1 or bounded synthesis~\cite{schewe2007bounded}
typically produce controllers as BDD or explicit state machine structures that
can be very large and difficult to inspect or simulate.

Regarding the STIMULUS tool, our approach was conceptually though of starting
from an opposite direction -- while STIMULUS essencially uses as central
formalism state machines wrapped by a syntactic-sugar English-like specification
language, \textsf{EARS-CTRL} uses a constrained version of the English language.
We have purposefully placed EARS at the center on our tool -- the goal has been
to adapt the subset of LTL used by \textsf{autoCode4} to EARS and to remain
unbiased towards the formalisms ``under the hood''. Unlike in our work, STIMULUS
relies on the state machines underlying the requirements to allow simulation as
in fact the approach's goal is to verify requirements and not to synthesize
usable controllers.\vspace{-.5cm}



\section{Limitations and Future Work}
\label{sec:limfuturework}
\vspace{-.3cm}Due to the boolean representation of sensors and actuators in
\textsf{autoCode4} it is currently not possible for \textsf{EARS-CTRL} to express or analyze
states of the system that involve numerical data. This naturally limits our
approach to controllers for systems where sensors or actuators can be
represented using boolean types.
For instance, expressing a state such as ``the throttle is pushed to 1/4 of its
capacity'' using \textsf{EARS-CTRL} would at best involve modelling four
different sensors in order to partition the input space of a single sensor. Such
an approach may prove to be infeasible and/or impractical and other code
synthesizers for \textsf{EARS-CTRL} might be considered in the future.\levi{which ones?}

Future work will concentrate on exploring the usability of
\textsf{EARS-CTRL} for larger case studies. In particular we expect to continue
the collaboration with Rolls-Royce in order obtain real requirements such that
the synthesis and verification mechanisms explained in this paper can be put to
the test in the field. It is of particular interest to understand not only how
\textsf{EARS-CTRL} \textsf{autoCode4}-based synthesis will scale, but also how
the verification and debug mechanisms we propose are helpful in practice.\levi{is this too self-deprecating?}


\input{text_paper/conclusion.tex}

%
% ---- Bibliography ----
%
\bibliographystyle{abbrv}
\bibliography{models_tool_2017}

\end{document}
